Dla indywidualnie wybranej klasy obrazów należało dobrać, zaimplementować i przetestować odpowiednie procedury wstępnego przetworzenia, segmentacji, wyznaczania cech oraz identyfikacji obrazów cyfrowych. Powstały w wyniku projektu program powinien poprawnie rozpoznawać wybrane obiekty dla reprezentatywnego zestawu obrazów wejściowych.

Wybrana klasa obrazów to zdjęcia zawierające logo mBanku (Rys. \ref{fig:logo_main}).

\begin{figure}[h]
    \centering
    \includegraphics[height=2.5cm]{figures/mBank_logo_main.png}
    \caption{Logo mBanku}
    \label{fig:logo_main}
\end{figure}